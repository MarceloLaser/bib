\newacronym{bnf}{BNF}{Backus-Naur Form}
\newacronym{corba}{CORBA}{Common Object Request Broker Architecture}
\newacronym{cbse}{CBSE}{Component-Based Software Engineering}
\newacronym{spl}{SPL}{Software Product Line}
\newacronym{sple}{SPLE}{Software Product Line Engineering}
\newacronym{sei}{SEI}{Software Engineering Institute}
\newglossaryentry{bindtime}
{
	name={binding time},
	description={The moment during which the properties of a software artifact are configured. Typically, these may be: compile-time, where the configuration is handled during the compilation process; link-time, where the configuration is handled at the moment of dependency solution and access to external libraries; load-time, where the configuration is handled the moment the software product is instanced (loaded), and; run-time, where the configuration is handled during the software's execution}
}
\newglossaryentry{grammar}
{
	name={grammar},
	description={The set of rules that define the structure of all possible strings within a language. For any one string (regardless of size) to belong to a particular language, it must match that language's grammar. Unless stated otherwise, this document uses the terms \emph{grammar} and \emph{context-free grammar} interchangeably. More information on context-free grammars and grammars in general can be found in \cite{HOPCROFT:2000}}
}
\newglossaryentry{regex}
{
	name={regular expression},
	description={A sequence of characters that represents a string pattern. A regular expression has several uses, among which the ones we are most interested in are searching a stream of text for matching strings and validating a string against a pre-determined pattern. More information on regular expressions can be found in \cite{HOPCROFT:2000}}
}
\newglossaryentry{variability}
{
	name={variability},
	description={From SEI \cite{BACHMANN:2005}: ``Variability is the ability of a system, an asset, or a development environment to support the production of a set of artifacts that differ from each other in a preplanned fashion.''}
}
\newglossaryentry{variant}
{
	name={variant},
	description={From SEI \cite{BACHMANN:2005}: ``The realization of a variable part, meaning the result of exercising the variation mechanism(s), is called a variant.''}
}
\newglossaryentry{connector}
{
	name={connector},
	description={From Taylor et al. \cite{TAYLOR:2009}: ``An architectural element tasked with effecting and regulating interactions among components.''}
}
\newglossaryentry{vpoint}
{
	name={variation point},
	description={Also known as variable part, it is a well-defined point in the SPL where modifications may be introduced \cite{BACHMANN:2005}.}
}
\newglossaryentry{paradigm}
{
	name={programming paradigm},
	description={From Van Roy \cite{VANROY:2009}: ``A programming paradigm is an approach to programming a computer based on a mathematical theory or a coherent set of principles.''}
}
